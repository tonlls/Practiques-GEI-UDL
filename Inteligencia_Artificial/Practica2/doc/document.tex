% !TeX spellcheck = ca
\documentclass{article}
\usepackage[catalan]{babel}
\usepackage[utf8]{inputenc}
\usepackage{graphicx}
\usepackage{tikz}
\usepackage{enumitem}
\usepackage{listings}
\usepackage{multirow}
\usepackage{float}
\usepackage{amsmath}
\usetikzlibrary{positioning,fit,calc,arrows.meta, shapes}
%\usepackage{bytefield}
\graphicspath{ {images/} }
\usepackage{array}


\newcolumntype{L}[1]{>{\raggedright\let\newline\\\arraybackslash\hspace{0pt}}m{#1}}
\newcolumntype{C}[1]{>{\centering\let\newline\\\arraybackslash\hspace{0pt}}m{#1}}
\newcolumntype{R}[1]{>{\raggedleft\let\newline\\\arraybackslash\hspace{0pt}}m{#1}}

%Tot això hauria d'anar en un pkg, però no sé com és fa
\newcommand*{\assignatura}[1]{\gdef\1assignatura{#1}}
\newcommand*{\grup}[1]{\gdef\3grup{#1}}
\newcommand*{\professorat}[1]{\gdef\4professorat{#1}}
\renewcommand{\title}[1]{\gdef\5title{#1}}
\renewcommand{\author}[1]{\gdef\6author{#1}}
\renewcommand{\date}[1]{\gdef\7date{#1}}
\renewcommand{\contentsname}{Índex}
\renewcommand{\maketitle}{ %fa el maketitle de nou
	\begin{titlepage}
		\raggedright{UNIVERSITAT DE LLEIDA \\
			Escola Politècnica Superior \\
			Grau en Enginyeria Informàtica\\
			\1assignatura\\
}
		\vspace{5cm}
		\centering\huge{\5title \\}
		\vspace{3cm}
		\large{\6author} \\
		\normalsize{\3grup}
		\vfill
		Professorat : \4professorat \\
		Data : \7date
\end{titlepage}}
%Emplenar a partir d'aquí per a fer el títol : no se com es fa el package
%S'han de renombrar totes, inclús date, si un camp es deixa en blanc no apareix

\renewcommand{\figurename}{Figura}
\renewcommand{\tablename}{Taula}
\title{Max-SAT}
\author{Sergi Simón Balcells}
\date{11 de Desembre 2019}
\assignatura{Intel·ligència Artificial}
\professorat{E. Torres}
\grup{GM3}

% Makes padding for table. Improves readability
\renewcommand{\arraystretch}{1.5}


\setlength{\tabcolsep}{18pt}
%Comença el document
\begin{document}
	\maketitle
	\newpage

\section{Introducció}
En aquesta pràctica es realitza la implementació de tres problemes amb l'eina de MAXSAT.
Aquests són 3 problemes dedicats a grafs, un problema de passar d'una fórmula no 13wm a 13wm, i
el problema de \textit{Software Package Upgrade} (SPU d'ara endavant). Aquest document tracta de dos
apartats:
\begin{itemize}
	\item L'implementació, que tractarà de les decisions preses en aquesta.
	\item Tests, que tractarà com s'ha avaluat el correcte funcionament del programa.
\end{itemize}
\section{Implementació}
En aquest apartat se seguirà les decisions que s'ha pres en la implementació.
Es començarà per l'apartat de 13wm, donat que la implementació d'aquest ha sigut
no ha influït a les implementacions dels altres apartats, mentre que els problemes de 
graf han influenciat a les del problema de SPU. Finalment, per preferència personal, 
s'ha utilitzat funcions d'alt ordre per estalviar-se el raonament. 
\subsection{13wm}
Per a fer el tractament de 13wm, s'ha procedit primer en tractar les clàusules soft, aquestes
s'han inclòs en la fórmula resultant, i s'han deixat les clàusules hard creades per a un nou mètode
de la fórmula, on es tradueix clàusules hard a 3. El mateix mètode s'ha utilitzat per afegir les clàusules
hard de l'última fórmula. Per comprovar el seu funcionament, es va solucionar el problema abans de passar
a 13wm i en passar-se. Es va considerar que si retornaven el mateix resultat, era que l'algorisme funcionava. Però, 
en haver-hi més d'un òptim podien retornar resultats diferents i continuar estant bé l'algorisme, pel que es va comprovar
amb el flag -n a -1, ja que aquest retorna tots els resultats.
\subsection{Problemes de graf}
Els tres problemes dels grafs s'han tractat com es diu en els apunts. Primer s'han realitzat les clàusules soft, 
i després les hard.
\subsubsection{Max Clique}
Per a la realització d'aquest problema es necessitava crear el conjunt d'arestes que no estaven
en el graf original. Per a aquesta realització, s'ha utilitzat la següent propietat:\\
\\
Sigui $E$ el conjunt d'arestes pertanyent al graf $G$, i $\mathcal{U}_E$ el conjunt d'arestes
possibles, llavors es compleix que $\mathcal{U}_E -E = \overline{E}$. \\
\\
Per a la realització d'aquest,
s'ha utilitzat un generador per a les arestes de $U_E$ i $\overline{E}$, ja que només és
necessari iterar-les un cop. Per altra banda, s'ha observat amb l'exemple de Petersen que les arestes
no estan particularment ordenades, pel que s'ha creat una estructura auxiliar que té les arestes ordenades.
A més a més, per aquesta estructura s'ha utilitzat un diccionari, ja que així el cost de mirar si una aresta està
al graf es torna, en el pitjor dels casos, $\mathcal{O}(1)$.

\subsection{SPU solver}
Per a realitzar l'SPU solver s'ha copiat fitxer dels problemes de graf, i 
s'ha adaptat tot el contingut al problema de SPU. Per aquest motiu, i 
per facilitat el debug, s'ha realitzat la visualització del problema dels paquets.
\subsubsection{Visualització}
La visualització s'ha dut a terme amb un digraf, on les arestes negres són les dependències i
les vermelles són els conflictes. La representació d'una dependència o una altra no
s'ha realitzat, ja que no s'ha trobat un sistema per a visualitzar-ho de forma clara 
sense sobrecarregar la visualització.
\subsubsection{SPU}
Per a realitzar l'SPU, s'ha continuat amb la forma de tractament dels grafs. Per a fer la traducció
del nom dels paquets a un nombre de la fòrmula, s'ha realitzat amb un \textit{defaultdict}, on el cas
estàndard és  la creació d'una variable. Per a fer la traducció s'ha invertit amb el codi:
\begin{lstlisting}[language=Python]
 inv_map = {v: k for k, v in generated.items()}                                                                                                                                                                                  
\end{lstlisting}
Ja que, si no, la traducció requereix d'un cost $\mathcal{O}(n^2)$, mentre que d'aquesta forma, en
el pitjor dels casos es requereix d'un cost $\mathcal{O}(n\log n)$.
\section{Tests}
La realització dels tests ha estat costosa, ja que s'han fet manualment. El  problema d'aquests és que un 
canvi en l'ordre de les clàusules d'una fórmula pot significar trobar un altre òptim, en el cas que n'hi hagi 
més d'un. Pels exemples de grafs s'ha realitzat a mà la solució dels tres grafs pels tres problemes. En executar-se
s'ha comprovat que l'òptim era el mateix que el trobat a mà.
Per a i13wm, s'ha acceptat que si la solució continua tenint el mateix òptim tant en el cas anterior com en el cas en 13wm, 
era que la solució plantejada estava bé. Finalment, per a comprovar que aquesta solució tenia sentit, s'ha escrit els problemes dels
grafs en dimacs i s'ha llegit comprovat que el 13wm donava una resposta vàlida en els tres. 
Per a SPU, només s'ha realitzat amb el test de \textit{sample}, ja que els altres eren més fàcil equivocar-se fent el resultat a mà.
\end{document}














